%i primi parametri sono relativi alle dimensioni del carattere
%ed alla tipologia di pagina. Potete scegliere tra
%'10pt', '11pt' and '12pt', per la pagina 'a4paper', 'letterpaper', 'a5paper', %'legalpaper', 'executivepaper' o 'landscape')
\documentclass[11pt,a4paper]{moderncv}

% lo stile del nostro curriculum
% come vedete ho scelto classic ma potete scegliere 
% 'moderncv' 'casual' (si default),'classic', 'oldstyle' o 'banking'
% e i colori 'blu', 'orange', 'green', 'red', 'purple', 'grey' o 'black'
\moderncvtheme[blue]{classic}

% alcuni pacchetti standard
\usepackage[italian]{babel}
\usepackage[utf8]{inputenc}
\usepackage[T1]{fontenc}
\usepackage{amsmath}
\usepackage[scale=0.8,top=1cm, bottom=2cm]{geometry}

% queste 2 righe servono per allargare la colonna di sinistra
\setlength{\hintscolumnwidth}{4cm}
\AtBeginDocument{\recomputelengths}

% altri pacchetti standard
%\usepackage{hyperref}
%\usepackage{url}

\usepackage{url}
\usepackage[bookmarks,pdftex,
pdfauthor={Francesco Soncina},
pdftitle={Curriculum Vitae},
pdfsubject={Curriculum Vitae},
pdfproducer={Latex with hyperref},
pdfcreator={pdflatex}]{hyperref}
\usepackage{booktabs}

\hypersetup{colorlinks = true,linkcolor=blue,urlcolor=blue}

% aggiungo la numerazione delle pagine (spesso inutile)
% obbligatorio se si usa hyperref (bug)
\pagenumbering{arabic}
\linespread{1,4}

% Dati personali
\firstname{Francesco}
\familyname{Soncina}
\title{Curriculum Vit\ae}
\address{Via Rillosi, 6}{25087, Salò (BS)}
\mobile{+39\,320\,0660220}
% la mail ed il sito web avranno il collegamento multimediale
% all'interno del pdf
\email{greensoncio@gmail.com}
%\homepage{www.ivanbortolin.it}
% io ho omesso la foto, ma se voi la volete inserire basta
%aggiungerla nella cartella e mettere il nome del file
%\photo[100pt][0.6pt]{nome_file} 


% da qui comincia il documento
\begin{document}
\maketitle

% per non far comparire i numeri di pagina
\pagestyle{empty}

% con il comando \sectio andiamo a creare delle sezioni nel documento e potremo 
% nominarle come più ci aggrada nonché spostarle all'interno del CV in
% base alle nostre esigenze e stile
\section{Informazioni Personali}
% per aggiungere voci nelle varie sezioni basta dare il comando
% \cvline {il_mio_testo}{il_mio_testo}
\cvline{nome}{Francesco}
\cvline{cognome}{Soncina}
\cvline{sesso}{maschile}
\cvline{luogo e data di nascita}{Gavardo (BS), 23 aprile 1990}
\cvline{nazionalità}{italiana}
\cvline{stato civile}{celibe}
\cvline{patente di guida}{cat. B, conseguita il 17 gennaio 2009}

\section{Titoli di Studio}

\cvline{marzo 2014}{\textbf{Laurea triennale}\newline
in Informatica presso l'Università degli Studi di Bologna con votazione 102/110}

\cvline{}{\textbf{Tesi di laurea} \cite{tesi}\newline
{\em IEEE 802.21: Media Independent Handover}\newline
Le competenze approfondite durante la stesura sono state: \newline
- architettura standard IEEE 802.21 \newline
- analisi dell'implementazione open-source "ODTONE"\newline
- realizzazione di un proxy simplex e full-duplex ad alta affidabilità}


\cvline{giugno 2010}{\textbf{Diploma di Maturità Scientifica}\newline
presso Liceo Scientifico Enrico Fermi a Salò con votazione 71/100}

\section{Accreditamenti}
\cvitem{maggio 2013}{\textbf{Test TECO}\newline Esito: 1225 (media nazionale: 1000)}
\cvitem{gennaio 2013} {\textbf{Borsa di Studio}\newline Borsa di studio per reddito erogata dall'azienda ER.GO}
\cvitem{novembre 2012} {\textbf{Idoneità Linguistica Inglese B1}\newline Idoneità linguistica di livello B1 conseguita presso il CILTA dell'Università di Bologna}
\cvitem{gennaio 2012} {\textbf{Borsa di Studio}\newline Borsa di studio per reddito erogata dall'azienda ER.GO}
\cvitem{gennaio 2011} {\textbf{Borsa di Studio}\newline Borsa di studio per reddito erogata dall'azienda ER.GO}

\section{Lingue Straniere}
\cvitem{}{\textbf{Inglese}\newline Scritto: ottimo\newline Parlato: buono}

\section{Conoscenze Informatiche}
\cvitem{\textbf{Linguaggi conosciuti}}{C, C++ (+ boost), Java (+ android, libgdx, gwt), Python, Bash, HTML5, CSS3, Javascript (+ jquery), SQL, \LaTeX (+ tikz)}
\cvitem{\textbf{Competenze acquisite}}{Amministrazione remota di servers GNU/Linux\newline Configurazione di servers Postgresql\newline Familiarità con terminali Bash\newline Sviluppo web con architettura REST\newline Sviluppo applicazioni mobili per Android con API native oppure con {\em LibGDX}\newline Sviluppo software in C secondo standard ANSI/POSIX\newline Sviluppo scripts Bash o Python\newline Sviluppo applicazioni Java native oppure con frameworks GWT, LibGDX, Android}

\cvitem{\textbf{Daily distro}}{Debian}

\section{Progetti}
\cvitem{marzo 2014} {\textbf{Realizzazione progetto di Tesi} \cite{progettotesi}\newline Sviluppo di un proxy ad alta affidabilità che possa utilizzare più interfacce di rete per comunicare in modo {\em simplex} oppure {\em full-duplex} con un altro {\em peer} [scritto in: C++ con libreria Boost]}

\cvitem{gennaio 2014} {\textbf{Realizzazione progetto del corso di "Sviluppo di Sistemi Virtuali"} \cite{vsd}\newline Sviluppo di un modulo per la macchina virtuale parziale UMView per il controllo della connettività di un singolo processo attraverso l'intercettazione delle {\em syscalls} di rete [scritto in: C]}

\cvitem{novembre 2013} {\textbf{Rilascio prima versione dell'applicazione Android {\em Trovatutto}} \cite{trovatutto} \newline Pubblicazione di un'applicazione per {\em smartphones} Android per la mappatura di esercizi commerciali e servizi pubblici di tutta Italia, dove qualsiasi utente può aggiungere istantaneamente nuovi punti di interesse visibili da tutti [scritto in: Java (Android), Python, SQL (postgresql)]}

\cvitem{settembre 2013} {\textbf{Realizzazione progetto del corso di "Sviluppo di Applicazioni Mobili"} \cite{shuttle}\newline Sviluppo di un videogioco single-player e multi-player 2D {\em cross-platform} per Android, iOS, PC e HTML5 attraverso il framework {\em LibGDX} [scritto in: Java (LibGDX)]}

\cvitem{luglio 2013} {\textbf{Realizzazione progetto del corso di "Ingegneria del Software"} \cite{lis}\newline Sviluppo di una piattaforma {\em browser-based} per la gestione di un sistema di noleggio automobili distribuito [scritto in: Java (GWT), SQL (postgresql)]}

\cvitem{settembre 2012} {\textbf{Realizzazione progetto del corso di "Sistemi Operativi"} \cite{so}\newline Sviluppo di un kernel minimale per l'architettura MIPS [scritto in: C]}

\cvitem{luglio 2012} {\textbf{Rilascio prima versione di {\em phoxy}} \cite{phoxy}\newline Pubblicazione di un proxy in grado di trasformare la comunicazione TCP tra un client ed un server in datagrammi UDP eventualmente cifrati con algoritmo a chiave simmetrica {\em AES256} [scritto in: C]}

\cvitem{maggio 2012} {\textbf{Realizzazione progetto del corso di "Reti di Calcolatori"} \cite{reti}\newline Sviluppo di un proxy in grado di trasformare un flusso TCP in datagrammi UDP, gestendo i pacchetti persi, doppi, fuori ordine e sfruttando tre canali di comunicazione [scritto in: C]}

\cvitem{febbraio 2012} {\textbf{Realizzazione progetto del corso di "Tecnologie Web"} \cite{progettotw} \cite{codicetw}\newline Sviluppo di un sito web basato su architettura REST per facilitare la geolocalizzazione di negozi e servizi pubblici di Bologna ed altre città [scritto in: HTML5, CSS3, Javascript (jquery), Python]}

\section{Esperienze Lavorative}
\cvitem{marzo 2014} {\textbf{Part-time studentesco}\newline Svolto presso la biblioteca centralizzata del Dipartimento di Agraria}
\cvitem{luglio 2012} {\textbf{Part-time studentesco}\newline Svolto presso il laboratorio del Dipartimento di Scienze dell'Informazione}

% voi aggiungete tutte le voci che ritenete opportune
% nel caso abbiate delle pubblicazioni, dopo aver creato l'opportuno file .bib
% NB: io ho disbilitato questi comandi
%\nocite{*}
\bibliographystyle{unsrt}
\bibliography{biblio.bib}    

% Ora ci resta solo che inserire la liberatoria per la privaci
% Io ho scannerizzato direttamente la mia firma e l'ho inserita come 
% immagine. Nel caso vogliate mettere dei puntini e firmare personalmente
% ogni CV dopo il "in fede" scrivere \makebox[9cm]{\dotfill}
\vspace{\fill}
{\footnotesize\noindent
Il sottoscritto è a conoscenza che, ai sensi dell'art. 26 della legge 15/68, le dichiarazioni mendaci, la falsità negli atti e l'uso di atti falsi sono puniti ai sensi del codice penale e delle leggi speciali ed autorizza al trattamento dei dati personali, in conformità alle disposizioni della legge sulla privacy (D.L.196/03). Inoltre, il sottoscritto ricorda che tutto il codice allegato è rilasciato sotto licenza GNU GPLv2 (o versioni successive) e le relative documentazioni sotto licenza CC BY-NC-SA, ove non specificato diversamente.}
\vspace*{0.8cm}\\
Bologna, \today\hfill In fede: \makebox[7cm] {\includegraphics[scale=.07]{firma}}
%\makebox[9cm]{\dotfill}
%per scalare la grandezza dell'immagine di firma, variate il parametro “scale”
\end{document}