%%%%%%%%%%%%%%%%%%%%%%%%%%%%%%%%%%%%%%%%%
% "ModernCV" CV and Cover Letter
% LaTeX Template
% Version 1.11 (19/6/14)
%
% This template has been downloaded from:
% http://www.LaTeXTemplates.com
%
% Original author:
% Xavier Danaux (xdanaux@gmail.com)
%
% License:
% CC BY-NC-SA 3.0 (http://creativecommons.org/licenses/by-nc-sa/3.0/)
%
% Important note:
% This template requires the moderncv.cls and .sty files to be in the same 
% directory as this .tex file. These files provide the resume style and themes 
% used for structuring the document.
%
%%%%%%%%%%%%%%%%%%%%%%%%%%%%%%%%%%%%%%%%%

%----------------------------------------------------------------------------------------
%	PACKAGES AND OTHER DOCUMENT CONFIGURATIONS
%----------------------------------------------------------------------------------------

\documentclass[11pt,a4paper,sans]{moderncv} % Font sizes: 10, 11, or 12; paper sizes: a4paper, letterpaper, a5paper, legalpaper, executivepaper or landscape; font families: sans or roman
\usepackage[italian]{babel}
\usepackage[utf8]{inputenc}
%\usepackage[T1]{fontenc}
\moderncvstyle{casual} % CV theme - options include: 'casual' (default), 'classic', 'oldstyle' and 'banking'
\moderncvcolor{blue} % CV color - options include: 'blue' (default), 'orange', 'green', 'red', 'purple', 'grey' and 'black'

\usepackage{lipsum} % Used for inserting dummy 'Lorem ipsum' text into the template

\usepackage[scale=0.75]{geometry} % Reduce document margins
%\setlength{\hintscolumnwidth}{3cm} % Uncomment to change the width of the dates column
%\setlength{\makecvtitlenamewidth}{10cm} % For the 'classic' style, uncomment to adjust the width of the space allocated to your name

%----------------------------------------------------------------------------------------
%	NAME AND CONTACT INFORMATION SECTION
%----------------------------------------------------------------------------------------

\firstname{Francesco} % Your first name
\familyname{Soncina} % Your last name

% All information in this block is optional, comment out any lines you don't need
\title{Curriculum Vit\ae}
\address{Via Rillosi, 6}{Salò, 25087, Italia}
\mobile{+39 320 0660220}
%\phone{(000) 111 1112}
%\fax{(000) 111 1113}
\email{greensoncio@gmail.com}
%\homepage{staff.org.edu/~jsmith}{staff.org.edu/$\sim$jsmith} % The first argument is the url for the clickable link, the second argument is the url displayed in the template - this allows special characters to be displayed such as the tilde in this example
%\extrainfo{additional information}
%\photo[70pt][0.4pt]{pictures/picture} % The first bracket is the picture height, the second is the thickness of the frame around the picture (0pt for no frame)
%\quote{"A witty and playful quotation" - John Smith}

%----------------------------------------------------------------------------------------

\begin{document}

\makecvtitle % Print the CV title

%----------------------------------------------------------------------------------------
%	EDUCATION SECTION
%----------------------------------------------------------------------------------------

\section{Istruzione}

\cventry{2010--2013}{Informatica}{}{Università di Bologna}{\textit{102}}{Laurea triennale}  % Arguments not required can be left empty
%\cventry{2007--2010}{Bachelor of Business Studies}{The University of California}{Berkeley}{\textit{GPA -- 7.5}}{Specialized in Commerce}

\section{Tesi di laurea}

\cvitem{Titolo}{\emph{IEEE 802.21: Media Independent Handover}}
\cvitem{Relatore}{Prof. Vittorio Ghini}
\cvitem{Materia}{Reti di calcolatori}
\cvitem{Descrizione}{Le competenze approfondite durante la stesura sono state: \begin{itemize}
		\item architettura standard IEEE 802.21
		\item analisi dell'implementazione open-source "ODTONE"
		\item realizzazione di un proxy {\em simplex} e {\em full-duplex} ad alta affidabilità
\end{itemize}}

%----------------------------------------------------------------------------------------
%	AWARDS SECTION
%----------------------------------------------------------------------------------------

\section{Accreditamenti}

\cvitem{maggio 2013}{Test TECO -- 1225}
\cvitem{gennaio 2013}{Borsa di Studio}
\cvitem{novembre 2012}{Inglese B1}
\cvitem{gennaio 2012}{Borsa di Studio}
\cvitem{gennaio 2011}{Borsa di Studio}


%----------------------------------------------------------------------------------------
%	LANGUAGES SECTION
%----------------------------------------------------------------------------------------

\section{Lingue}

\cvitemwithcomment{Italiano}{Madrelingua}{}
\cvitemwithcomment{English}{Intermedio}{scritto ottimo, parlato buono}
%\cvitemwithcomment{Spanish}{Intermediate}{Conversationally fluent}
%\cvitemwithcomment{Dutch}{Basic}{Basic words and phrases only}


%----------------------------------------------------------------------------------------
%	WORK EXPERIENCE SECTION
%----------------------------------------------------------------------------------------

\section{Esperienze lavorative}

%\subsection{Vocational}

\cventry{luglio 2014 -- maggio 2015}{IT Consultant}{\textsc{ALTEN}}{Bologna}{presso UniCredit Group}{
\begin{itemize}
\item Analisi
	\begin{itemize}
	\item Raccolta e formalizzazione dei requisiti e vincoli
	\item Analisi dei requisiti e vincoli
	\item Analisi approfondita delle specifiche e formalizzazione logica dei processi ed entità coinvolte con produzione di diagrammi E/R e specifiche workflow
	\end{itemize}
\item Progettazione
	\begin{itemize}
	\item Realizzazione del documento di specifica del design
	\item Realizzazione mockups e stime effort
	\end{itemize}
\item Sviluppo
	\begin{itemize}
	\item Sviluppo iniziale della soluzione proposta: (successivamente sviluppato in outsourcing)
Progettazione di una soluzione basata su stack JavaEE + Struts2 + Hibernate, con particolare attenzione a rimanere indipendenti dall’application server e dal database utilizzato
	\end{itemize}
\item Funzionale
	\begin{itemize}
	\item Supporto allo sviluppo per i processi (sviluppato in outsourcing)
	\item Verifiche sull’andamento dei lavori
	\item Validazione soluzione realizzata
	\end{itemize}
\item UAT:
	\begin{itemize}
	\item Realizzazione del documento di specifica dell’User Acceptance Test, ovvero scrittura dei casi d’uso da testare in fase di collaudo, il quale esito formalizza l’accettazione da parte del cliente della soluzione realizzata
	\end{itemize}
\item Collaudi:
	\begin{itemize}
	\item Partecipazione alle sessioni di collaudo al fine di identificare le anomalie e segnalarle al team di sviluppo arricchendo le segnalazioni con dettagli tecnici
	\end{itemize}
\item Testing / QA
	\begin{itemize}
	\item Creazione segnalazioni
	\item Classificazione e monitoraggio segnalazioni
	\item Supporto risoluzioni anomalie
	\item Testing delle correzioni effettuate
	\item Promozione builds e gestione richieste di deploy
	\end{itemize}
\item Customer Support
	\begin{itemize}
	\item Gestione ed assegnazione delle segnalazioni
	\end{itemize}
\end{itemize}}


%------------------------------------------------

%\cventry{2010--2011}{Summer Intern}{\textsc{Lehman Brothers}}{Los Angeles}{}{Rated "truly distinctive" for Analytical Skills and Teamwork.}

%------------------------------------------------

\subsection{Miscellanea}

\cventry{marzo 2014}{Part-time studentesco}{Università di Bologna}{}{}{}
\cventry{luglio 2012}{Part-time studentesco}{Università di Bologna}{}{}{}


%----------------------------------------------------------------------------------------
%	COMPUTER SKILLS SECTION
%----------------------------------------------------------------------------------------

\section{Competenze}

\cvitem{Skills}{GNU/Linux SysAdmin, DevOps, IT Automation}
\cvitem{Linguaggi}{Bash, JavaScript, C, C++, Java, Python, SQL, HTML5, CSS3, \LaTeX}
\cvitem{Applicativi}{Nginx, PostgreSQL, NodeJS, MySQL, Apache}
\cvitem{DVCS}{Git}
\cvitem{Provisioning}{Vagrant, Ansible}
\cvitem{Altro}{ViM, Tmux}
\cvitem{Daily distro}{Debian}


\section{Progetti}
\cvitem{marzo 2014} {\textbf{Realizzazione progetto di Tesi} \newline Sviluppo di un proxy ad alta affidabilità che possa utilizzare più interfacce di rete per comunicare in modo {\em simplex} oppure {\em full-duplex} con un altro {\em peer} [scritto in: C++ con libreria Boost]}

\cvitem{gennaio 2014} {\textbf{Realizzazione progetto del corso di "Sviluppo di Sistemi Virtuali"} \newline Sviluppo di un modulo per la macchina virtuale parziale UMView per il controllo della connettività di un singolo processo attraverso l'intercettazione delle {\em syscalls} di rete [scritto in: C]}

\cvitem{novembre 2013} {\textbf{Rilascio prima versione dell'applicazione Android {\em Trovatutto}} \newline Pubblicazione di un'applicazione per {\em smartphones} Android per la mappatura di esercizi commerciali e servizi pubblici di tutta Italia, dove qualsiasi utente può aggiungere istantaneamente nuovi punti di interesse visibili da tutti [scritto in: Java (Android), Python, SQL (PostgreSQL)]}

\cvitem{settembre 2013} {\textbf{Realizzazione progetto del corso di "Sviluppo di Applicazioni Mobili"} \newline Sviluppo di un videogioco single-player e multi-player 2D {\em cross-platform} per Android, iOS, PC e HTML5 attraverso il framework {\em LibGDX} [scritto in: Java (LibGDX)]}

\cvitem{luglio 2013} {\textbf{Realizzazione progetto del corso di "Ingegneria del Software"} \newline Sviluppo di una piattaforma {\em browser-based} per la gestione di un sistema di noleggio automobili distribuito [scritto in: Java (GWT), SQL (postgresql)]}

\cvitem{settembre 2012} {\textbf{Realizzazione progetto del corso di "Sistemi Operativi"} \newline Sviluppo di un kernel minimale per l'architettura MIPS [scritto in: C]}

\cvitem{luglio 2012} {\textbf{Rilascio prima versione di {\em phoxy}} \newline Pubblicazione di un proxy in grado di trasformare la comunicazione TCP tra un client ed un server in datagrammi UDP eventualmente cifrati con algoritmo a chiave simmetrica {\em AES256} [scritto in: C]}

\cvitem{maggio 2012} {\textbf{Realizzazione progetto del corso di "Reti di Calcolatori"}\newline Sviluppo di un proxy in grado di trasformare un flusso TCP in datagrammi UDP, gestendo i pacchetti persi, doppi, fuori ordine e sfruttando tre canali di comunicazione [scritto in: C]}

\cvitem{febbraio 2012} {\textbf{Realizzazione progetto del corso di "Tecnologie Web"}\newline Sviluppo di un sito web basato su architettura REST per facilitare la geolocalizzazione di negozi e servizi pubblici di Bologna ed altre città [scritto in: HTML5, CSS3, Javascript (jQuery), Python]}


%----------------------------------------------------------------------------------------
%	COMMUNICATION SKILLS SECTION
%----------------------------------------------------------------------------------------

%\section{Communication Skills}

%\cvitem{2010}{Oral Presentation at the California Business Conference}
%\cvitem{2009}{Poster at the Annual Business Conference in Oregon}


%----------------------------------------------------------------------------------------
%	INTERESTS SECTION
%----------------------------------------------------------------------------------------

\section{Interessi}

\renewcommand{\listitemsymbol}{-~} % Changes the symbol used for lists

%\cvlistdoubleitem{Crittografia}{Crittografia}
%\cvlistdoubleitem{Forensica}{Dancing}
\cvlistitem{Sicurezza informatica}
\cvlistitem{Crittografia}
\cvlistitem{Forensica}
\cvlistitem{Prototipazione rapida}
\cvlistitem{IT Automation}
\cvlistitem{Continous Integration}

%----------------------------------------------------------------------------------------
%	COVER LETTER
%----------------------------------------------------------------------------------------

% To remove the cover letter, comment out this entire block

%\clearpage

%\recipient{Risorse Umane}{Tesla Consulting\\Piazza dei Colori, 26\\Bologna 40138} % Letter recipient
%\date{\today} % Letter date
%\opening{Alla cortese attenzione del reparto HR} % Opening greeting
%\closing{Distinti saluti,} % Closing phrase
%\enclosure[In allegato]{curriculum vit\ae{}} % List of enclosed documents

%\makelettertitle % Print letter title

%Ho deciso di candidarmi per questa posizione lavorativa in quanto seguo Tesla Consulting da parecchio tempo, avendola conosciuta in occasione della prima edizione de {\em Hackinbo} e successivamente grazie ai servizi girati per il programma televisivo {\em Le Iene}. Sono da sempre affascinato ed interessato alla sicurezza informatica e sono convinto di poter crescere ed imparare molto da una esperienza come questa.

%\makeletterclosing % Print letter signature

%----------------------------------------------------------------------------------------

\end{document}